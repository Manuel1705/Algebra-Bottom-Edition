\section{Divisibilità}
\subsection{ Divisori e Multipli}
\bottom{
    Sia \((s, \cdot)\) un semigruppo commutativo. Allora
    $$\forall x, y \in s$$
    $$\substack{x | y \\
    x \text{ divide } y\\
    y \text{ e' multiplo di } x
    } :\iff \exists z \in s\ (xz = y)$$
    $$DIV_s(x) = \{y \in s \mid y | x\}$$
    $$MULT_s(x) = \{z \in s \mid x | z\}$$
}

\subsection{ Elementi Associati}
\bottom{
    Sia \((s, \cdot)\) un semigruppo commutativo. Allora diremo che due elementi \(x, y\) sono associati se si dividono a vicenda.

    $$x, y \text{ associati } :\iff x \in DIV_s(y) \land y \in DIV_s(x)$$
    Definiamo l'insieme degli associati di \(x\) come:
    $$ASSOC_s(x) = \{y \in s \mid x|y \land y|x\}$$
}

\subsection{ Associati di un Elemento Cancellabile}
\bottom{
    Sia \((s, \cdot)\) un monoide commutativo. Se \(x \in s\) è un elemento cancellabile, allora:

    $$ASSOC_s(x) = \{ xu \in s \mid u \in U(s)\}$$

    Cioè tutti gli associati di un elemento cancellabile $x$ sono il prodotto di \(x\) per un elemento invertibile di \(s\).
}
\bottomp{
    \(\supseteq)\) Sia \(u \in U(s)\). Allora
    \begin{align*}
        &&&x|xu\\
        &\text{ma}\\
        &&&xuu^{-1} = x\\
        &\text{quindi}\\
        &&&xu|x\\
        &\text{Pertanto}\\
        &&&xu \in ASSOC_s(x)
    \end{align*}

    \(\subseteq)\) Sia \(y \in ASSOC_s(x)\).
    Allora $$\exists w,z\in s$$
    $$y = xw$$
    $$x = yz$$ 
    Allora, usando la cancellabilità di \(x\):
    $$x = yz = xwz \implies wz = 1 \implies w, z \text{ invertibili } \implies$$
    $$\implies w \in U(s) \implies y \in \{xu \mid u \in U(s)\}$$

    Da cui la tesi.
}

\subsection{ Associati hanno stessi Divisori e Multipli}
\bottom{
    Sia \((s, \cdot)\) un monoide commutativo. Allora, \(\forall x, y \in s\):
    $$y \in ASSOC_s(x) \stackrel{\text{(1)}}{\iff} DIV_s(x) = DIV_s(y) \stackrel{\text{(2)}}{\iff} MULT_s(x) = MULT_s(y)$$
}
\bottomp{
    \(1 \Rightarrow )\) Segue dalla transitività della divisione \(|\).

    \(1 \Leftarrow)\) \(y \in DIV(y) \implies y \in DIV(x) \implies y | x.\) 

    Lo stesso vale per \(x\), quindi \(y \in ASSOC(x)\).

    \(2 \Leftrightarrow)\) Segue immediatamente dalla definizione di insieme dei divisori e dei multipli.
}

\subsection{ Massimi Comuni Divisori e Minimi Comuni Multipli}
\bottom{
    Sia \((s, \cdot)\) un monoide commutativo e sia \(t \subseteq s\). Allora definiamo:
    $$MCD_s(t) = \{d \in \bigcap_{x \in t} DIV_s(x) \mid \forall z \in \bigcap_{x \in t} DIV_s(x)\ (z | d)\}$$
    $$mcm_s(t) = \{d \in \bigcap_{x \in t} MULT_s(x) \mid \forall z \in \bigcap_{x \in t} MULT_s(x)\ (d | z)\}$$
    Bisogna fare attenzione che nonostante si utilizzino i termini "massimo" e "minimo" questi non vanno intesi come terminologia delle relazioni d'ordine, in quanto la divisione non è necessariamente una relazione d'ordine. Il massimo e i minimi di un insieme sono sempre unici, mentre MCD ed mcm, come si può vedere dalla definizione, sono insiemi e pertanto ne possono esistere molteplici.
}

\subsection{ Divisori Banali}
\bottom{
    Sia \((s, \cdot)\) un monoide commutativo e sia \(x \in s\). Allora definiamo l'insieme dei divisori banali:

    $$BDIV_s(x) = U(s) \cup ASSOC_s(x)$$
}

\subsection{ Elemento Irriducibile (in un Dominio di Integrità)}
\bottom{
    Sia \((s, +, \cdot)\) un dominio di integrità e sia \(x \in s\). Allora \(x\) si dice:

    $$x \text{ irriducibile } :\iff x \notin U(s) \land DIV_s(x) = BDIV_s(x)$$
}

\subsection{ Primo}
\bottom{
    Sia \((s, \cdot)\) un monoide commutativo.

    $$p \in s \text{ primo } :\iff \forall a, b \in s\ (p|ab \implies p|a \lor p|b)$$
}

\subsection{ Coprimi}
\bottom{
    Sia \((s, \cdot, 1_s)\) un monoide commutativo. Allora:
    $$x,y \in s \text{ coprimi } :\iff 1_s \in MCD(\{x, y\})$$

    Cioè se l'unità è un loro MCD.
}

\subsection{ Monoide Cancellativo}
\bottom{
    Un monoide commutativo si dice cancellativo se ogni suo elemento è cancellabile.
}

\subsection{ Monoide Fattoriale}
\bottom{
    Un monoide commutativo \((m, \cdot)\) si dice fattoriale se vale una delle seguenti proprietà:
    \begin{enumerate}
        \item Ogni \(x \in m - U(m)\) è un prodotto di primi.
        \item Ogni \(x \in m - U(m)\) è prodotto di irriducibili, ed ogni irriducibile è primo.
        \item Ogni \(x \in m - U(m)\) è prodotto di irriducibili, ed ogni fattorizzazione è unica a meno dell'ordine dei fattori e del prodotto per invertibili.
    \end{enumerate}

    Si dimostra che queste tre condizioni sono fra loro equivalenti.
}

\subsection{ Anello Fattoriale}
\bottom{
    Un anello commutativo unitario \((a, +, \cdot)\) si dice anello fattoriale se \((a - \{0_a\}, \cdot)\) è un monoide fattoriale.
}

\subsection{ Caratterizzazione di MCD e mcm per Associati}
\bottom{
    Sia \((s, \cdot)\) un monoide commutativo, siano \(x, y \in s\). allora:
    $$m \in MCD(x,y) \iff ASSOC(m) = MCD(x, y)$$
    $$m \in mcm(x, y) \iff ASSOC(m) = mcm(x, y)$$
}
\bottomp{
    \(\rightarrow)\) $$m|x \land m|y\ \land\ \forall z \in s\ (z|x \land z|y \implies z|m)$$ poiché esso è MCD.
    Sia $$n \in ASSOC(m) \implies n|m \land m|n \implies n|x \land n|y\ \land\ \forall z \in s\ (z|m \implies z|n)$$
    $$\implies n|x \land n|y\ \land\ \forall z \in s\ (z|x \land z|y \implies z|n) \implies n \in MCD(x,y)$$
    Quindi $$ASSOC(m) \subseteq MCD(x,y)$$
    Se invece $$n \in MCD(x,y) \implies m|n \land n|m \implies n \in ASSOC(m)$$
    Pertanto $$MCD(x,y) \subseteq ASSOC(m) \implies ASSOC(m) = MCD(x,y)$$

    \(\leftarrow)\) \(m \in ASSOC(m)\) in quanto \((s, \cdot)\) è un monoide. $$ASSOC(m) = MCD(x,y) \implies m \in MCD(x,y)$$
}

\subsection{ Fattorizzazione in Primi in un Monoide Fattoriale}
\bottom{
    Sia \((m, \cdot)\) un monoide fattoriale. Sia \(a \in m - U(m)\), allora $$a = p_1^{k_1} \cdot p_2^{k_2} \cdot ... \cdot p_n^{k_n}$$
    I divisori di \(a\) sono tutti e soli gli elementi associati ad elementi del tipo $$p_1^{l_1} \cdot ... \cdot p_n^{l_n}\text{ con } 0 \leq l_i \leq k_i,\ \forall i \in I_n$$
}

\subsection{ Numero di Divisori in \(\mathbb N\)}
\bottom{
    Corollario di "Fattorizzazione in Primi in un Monoide Fattoriale"

    Sia $$a = p_1^{k_1} \cdot p_2^{k_2} \cdot ... \cdot p_n^{k_n} \in \mathbb N$$ allora \(a\) ha esattamente $$\prod_{i = 1}^n (k_i + 1)$$ divisori.
}

\subsection{ Numero di Divisori in \(\mathbb Z\)}
\bottom{
    Corollario di "Fattorizzazione in Primi in un Monoide Fattoriale"

    Sia $$a = p_1^{k_1} \cdot p_2^{k_2} \cdot ... \cdot p_n^{k_n} \in \mathbb Z$$ allora \(a\) ha esattamente $$2 \cdot \prod_{i = 1}^n (k_i + 1)$$ divisori.
}

\subsection{ MCD e mcm dalle Fattorizzazioni}
\bottom{
    Sia \((m, \cdot)\) un monoide fattoriale e siano:
    $$a = p_1^{k_1} \cdot ... \cdot p_n^{k_n}$$
    $$b = p_1^{l_1} \cdot ... \cdot p_n^{l_n}$$
    Dove \(p_1, ..., p_n\) comprende tutti i primi che compaiano nelle fattorizzazioni di \(a\) o \(b\). Nel caso un primo appaia nella fattorizzazione di \(a\), ma non di \(b\) (o viceversa), allora esso può essere considerato come elevato allo zero nella fattorizzazione in cui non compare.

    Poniamo inoltre 
    $$\forall i \in \mathbb N\ (1 \leq i \leq n \implies \alpha_i = MAX(k_i, l_i) \land \beta_i = MIN(k_i, l_i))$$
    Allora, abbiamo che:
    $$m = p_1^{\alpha_1} \cdot ... \cdot p_n^{\alpha_n} \in mcm(a, b)$$
    $$M = p_1^{\beta_1} \cdot ... \cdot p_n^{\beta_n} \in MCD(a, b)$$

    ESEMPIO:
    \begin{align*}
        a &= 2 \cdot 3^2 \cdot 5 &&= 90 = 2 \cdot 3^2 \cdot 5 \cdot 11^0 \cdot 13^0 \\
        b &= 3 \cdot 5^2 \cdot 11 \cdot 13 &&= 10725= 2^0 \cdot 3 \cdot 5^2 \cdot 11 \cdot 13 
    \end{align*}
    Pertanto:
    \begin{align*}
        &m = 2 \cdot 3^2 \cdot 5^2 \cdot 11 \cdot 13 &&=  64350 \\
        &M =  2^0 \cdot 3 \cdot 5 \cdot 11^0 \cdot 13^0 &&= 15 
    \end{align*}
}

\subsection{ Associati da MCD e mcm}
\bottom{
    Corollario di "Teorema: MCD e mcm dalle Fattorizzazioni"

    Se $$m \in mcm(a, b) \land M \in MCD(a, b)$$ allora $$m\cdot M \in ASSOC(a, b)$$
}

\subsection{ Proprietà di Divisione Lineare dei Divisori Comuni}
\bottom{
    Sia \((s, +, \cdot)\) un anello commutativo unitario e siano \(a, b \in s\). Se $$d \in DIV_{(s, \cdot)}(a) \cap DIV_{(s, \cdot)}(b)$$ allora $$\forall x, y \in s\ (d|(xa + yb))$$
    Cioè i divisori comuni dividono ogni combinazione lineare degli elementi.
}
\bottomp{
    $$d|a \land d|b \implies (\exists h, k \in \mathbb Z)(a = dk \land b = dh)$$
    E quindi:
    $$\forall x, y \in Z\ (ax + by = dkx + dhy = d(kx + hy) \implies d|(ax + by))$$
}

\subsection{ Valore Assoluto}
\bottom{
    Sia \(n \in \mathbb Z\). Allora definiamo la funzione valore assoluto come:

    $$|n|: n \in \mathbb Z \mapsto 
    \begin{cases}
        n & \text{ se } n \in \mathbb N \\
        -n & \text{ se } n \in \mathbb Z - \mathbb N
    \end{cases}$$
}

\subsection{ Teorema della Divisione Euclidea (o con Resto)}
\bottom{
    $$\forall m, n \in \mathbb Z\ (m \neq 0 \implies (\exists! (q,r) \in \mathbb Z \times \mathbb N)(n = mq + r \land 0 \leq r < |m|))$$
}
\bottomp{
    (Caso $n\in \mathbb N$)

    Dimostriamo per induzione di seconda forma su \(n\).

    Allora se \(n = 0\), scelgo \(q = 0 = r\).

    Se \(0 < n < |m|\), allora \(q = 0, r = n\).

    Se \(n = |m|\), allora ci sono due possibilità: $$n = m \text{ e } q = 1, r = 0$$ oppure $$n = -m \text{ e } q = -1, r = 0$$

    Se \(n > |m|\). Consideriamo vero $\forall i\in \mathbb N\ |m|\le i<n$. Allora prendiamo \(n - |m| < n\), e quindi per ipotesi induttiva $$\exists q_1, r_1\ (n - |m| = mq_1 + r_1 \land 0 \leq r_1 \leq |m|)$$
    Quindi \(n = mq_1 + |m| + r_1\).

    Quindi, se \(m > 0\), allora \((q, r) = (q_1 + 1, r_1)\).

    Se \(m < 0\), allora \((q, r) = (q_1 - 1, r_1)\).

    Allora per induzione la tesi vale \(\forall n \in \mathbb N\).

    (Caso $n\in\mathbb Z-\mathbb N$)
    sappiamo che la tesi vale \(\forall n \in \mathbb N\).
    Supponiamo invece che \(n \in \mathbb Z - \mathbb N\). Allora $$-n \in \mathbb N \implies -n = mq_1 + r_1$$ con $$0 \leq r < |m| \implies n = m(-q_1) - r_1 = m(-q_1) - r_1 + |m| - |m|$$
    (aggiungo $+ |m| - |m|$ cosi che $0\le r<|m|$)\\
    Consideriamo i due casi possibili:
    $$m > 0 \implies n = m(-q_1 - 1) + m - r \implies (q, r) = (-q_1 - 1, m - r)$$
    $$m < 0 \implies n = m(-q_1 + 1) - m - r \implies (q, r) = (-q_1 + 1, -m - r)$$
    Pertanto la tesi è valida anche in \(\mathbb Z\).

    Dimostrazione che \((q, r)\) è unico:
    Siano \((q_1, r_2), (q_2, r_2) \in \mathbb Z \times \mathbb N\). Ipotizziamo WLOG che \(0 \leq r_1 \leq r_2 < |m|\) e che \(n = mq_1 + r_1 = mq_2 + r_2\).
    Allora $$m(q_1 - q_2) = r_2 - r_1$$ e quindi $$|m||q_1 - q_2| = |m(q_1 - q_2)| = |r_2 - r_1|\text{ e } 0 \leq |r_2 - r_1| < |m|$$
    Segue che \(|m||q_1 - q_2| < |m|\) che può succedere solo, dato che \(m \neq 0\) per ipotesi, se \(|q_1 - q_2| = 0 \implies q_1 = q_2\).

    E quindi $$n = mq_1 + r_1 = mq_1 + r_2 \implies r_1 = r_2$$ da cui la tesi.
}

\subsection{ Algoritmo delle Divisioni Successive}
\bottom{
    \includegraphics*[scale=.2]{img/ADS.png}

    Siamo sicuri che l'algoritmo abbia fine perché la successione dei resti è crescente limitata inferiormente dallo zero.
    \begin{align*}
        a &= bq_1 + r_1 \\
        b &= r_1q_2 + r_2 \\
        r_1 &= r_2q_3 + r_3\\
        &\dots\\
        r_{t-4} &= r_{t-3}q_{t-2} + r_{t-2}\\
        r_{t-3} &= r_{t-2}q_{t-1} + r_{t-1}\\
        r_{t-2} &= r_{t-1}q_t + r_t
    \end{align*}
    con $r_t=0$\\
    $MCD(a,b)=\pm r_{t-1}$\\    
    Con i resti:
    \(0 = r_t < r_{t-1} < \dots < r_1 < b\)
}

\subsection{ Teorema di Bézout}
\bottom{
    $$\forall (a, b) \in \mathbb Z \times \mathbb Z - \{(0, 0)\}$$
    $$\forall d \in MCD(a, b)$$
    $$\exists u, v \in \mathbb Z$$
    $$(d = au + bv)$$
    Cioè, per ogni MCD di una coppia di valori \(a, b\), esiste una combinazione lineare dei due che lo esprime.
}
\bottomp{
    Si consideri l'Algoritmo delle Divisioni Successive, per equazioni. Sia \(t\) il minimo numero di passi tali che \(r_t = 0\). Se \(t = 1\), allora \(r_1 = 0\) e \(a = bq_1 \implies b \in MCD(a, b)\) e \(b = a \cdot 0 + b \cdot 1\). Quindi \((u, v) = (0, 1)\).

    Se \(t = 2\), allora \(r_1 \neq 0 \land r_2 = 0\). Quindi \(r_1 \in MCD(a, b)\) e \(r_1 = a \cdot 1 + b \cdot (-q_1)\). Quindi \((u,v) = (1, -q_1)\).
    
    Supponiamo vero l'asserto per ogni \(r_i: 1 \leq i < t\). Dato che $$r_t = 0 \implies r_{t-1} \in MCD(a, b)$$ 
    $$r_{t-1} = r_{t-3} + r_{t-2}(-q_{t-1})$$
    perchè
    $r_{t-3}=r_{t_2}q_{t-1}+r_{t-1}$\\
     Ma per l'ipotesi induttiva, allora $$\exists u, v, w, x \in \mathbb Z\ (r_{t-3} = au + bv \land r_{t-2} = aw + bx)$$
    Quindi $$r_{t-1} = (au + bv) + (aw + bx)(-q_{t-1}) =$$
    $$= aw + bw - awq_{t-1} - bxq_{t-1} = a(w-wq_{t-1}) + b(v - xq_{t-1})$$
    
    Da cui la tesi.
}

\subsection{ Lemma di Euclide}
\bottom{
    Siano \(a, b, c \in \mathbb Z\). Se \(a, b\) sono coprimi, allora \(a | bc \implies a | c\).
}
\bottomp{
    \(1 \in MCD(a, b)\) perché sono coprimi. Per il Teorema di Bézout, $$\exists u, v \in \mathbb Z\ (au + bv = 1)$$
     Quindi \(c = acu + bcv\). Dato che \(a | ac\) (banalmente) e \(a | bc\) (per ipotesi), allora $$\exists h, k \in \mathbb Z\ (c =  acu + bcv = ahu + akv \implies c = a(hu+kv) \implies a | c)$$
     ($ac=ah$ perchè $a|ac$ quindi $\exists h\in \mathbb Z (ah=ac)$)\\
     ($bc=ak$ perchè $a|bc$ per ipotesi quindi $\exists k\in \mathbb Z (ak=bc)$)
}

\subsection{ In $\mathbb Z$ i primi sono tutti e soli gli irriducibili}
\bottom{
    In \(\mathbb Z\), i primi sono tutti e soli gli irriducibili.
    $$p \in \mathbb Z \text{ primo} \iff p \text{ irriducibile}$$
}
\bottomp{
    \(\Rightarrow)\) Sia \(p \in \mathbb Z\) primo e siano \(a, b \in \mathbb Z\) tali che \(p = ab\).\\
    \(p\) è primo \(\implies p | a \lor p|b\).\\
    Ipotizziamo senza ledere la generallità che \(p|a\).\\
    Quindi $$p | a \land a | p^* \implies p \in ASSOC(a) = \{a, -a\}$$
    Quindi $$p = \pm a \implies b = \pm 1$$
    Dunque, \(p\) ha solo divisori banali ed è dunque irriducibile.

    \(\Leftarrow)\) Sia \(p \in \mathbb Z\) irriducibile, cioè \(DIV(p) = BDIV(p) = \{-1, 1, p, -p\}\).\\
    Siano \(a, b \in \mathbb Z\) tali che \(p|ab\).\\
    Supponendo che \(p \nmid a\), bisogna dimostrare che necessariamente \(p | b\).\\
    Si ha che $$MCD(p, a) \subseteq DIV(p) = \{- 1, 1, -p, p\}^{**}$$
    ma dato che \(p \nmid a\), allora \(MCD = \{-1, 1\}\) e quindi i due sono coprimi.\\
    Per il Lemma di Euclide, \(p | b\).\\\\
     $(^*)\ a|p$ perchè $p=ab$ per ipotesi quindi $\exists b\in \mathbb Z\ (ab=p)$.\\
    $(^{**})$ dato che l'$MCD\subseteq DIV(p)\cap DIV(a)$.
}

\subsection{ Caratterizzazione Lineare di Classe di Resto}
\bottom{
    $$[a]_m = \{b \in \mathbb Z \mid m|(a-b)\} = \{a + km \mid k \in \mathbb Z\}$$
    Questo perchè sommando e sottraendo più volte $m$ al rappresentante della classe possiamo ottenere tutti gli elementi della classe.
}

\subsection{ Operazione Parziale: Modulo}
\bottom{
    $$\forall (a, m) \in \mathbb (Z \times (\mathbb Z - \{0\}))\ (a \bmod m = MIN([a]_m \cap \mathbb N))$$

    Il modulo è un'operazione "parziale" perché non è definita in \(\mathbb Z \times \mathbb Z\), ma in \(\mathbb Z \times (\mathbb Z - \{0\})\), cioè non è possibile effettuare \(a \bmod 0\).

    \subsubsection*{ Proprietà del Modulo}
    \bottom{
        \begin{itemize}
            \item \(a \bmod m < |m|\)
            \item \(a \bmod m = \text{ resto di } DE(a, m)\)
        \end{itemize}
    }

    \subsubsection*{ Notazioni del Modulo}
    \bottom{
        Le seguenti notazioni sono equivalenti:
        \begin{itemize}
            \item \(a \bmod m\)
            \item \(a \text{ \% } m\)
            \item \(REST(a, m)\)
        \end{itemize}
    }
}

\subsection{ Caratterizzazione di \(\mathbb Z_m\)}
\bottom{
    Sia \(m \in \mathbb N - \{0\}\). Allora \(\mathbb Z_m = \{[0]_m, [1]_m, \dots, [m-1]_m\}\) e in particolare si ha che \(|\mathbb Z_m| = m\)
}
\bottomp{
    L'insieme \(\{[0]_m, [1]_m, \dots, [m-1]_m\}\) è un insieme di classi di equivalenza di elementi di \(\mathbb Z\), quindi per definizione dev'essere sottoinsieme del suo insieme delle classi di resto.
    Sia \(a \in \mathbb Z\). $$DE(a, m) = (q, r) \land a = qm + r \land 0 \leq r < |m|$$ 
    Quindi $$qm = a - r\implies m|a-r\implies [a]_m = [r]_m$$
    Quindi $$\mathbb Z_m \subseteq \{[0]_m, [1]_m, \dots, [m-1]_m\}$$
    Quindi, dato che si contengono a vicenda, per estensionalità $$\mathbb Z_m = \{[0]_m, [1]_m, \dots, [m-1]_m\}$$

    Vogliamo adesso dimostrare che le classi di resto in \(\mathbb Z_m\) sono a due a due distinte, e sono quindi \(m\) in numero.
    Siano $$i, j \in \mathbb Z: 0 \leq i \leq j < m \land [i]_m = [j]_m$$ Allora $$0 \leq j - i < m \land (\exists k \in \mathbb Z)(j = i + km) \implies j - i = km \implies j - i = 0$$ in quanto esso è strettamente minore di m.
}

\subsection{ Relazione di Equivalenza Compatibile}
\bottom{
    Sia \(s \neq \emptyset\) e \(*\) un'operazione binaria interna ad esso, e \(\sim\) una relazione di equivalenza su esso. Allora:

    $$\sim \text{ compatibile a sx in } (s, *) :\iff \forall a, b, c \in s\ (a \sim b \implies c*a \sim c*b)$$
    $$\sim \text{ compatibile a dx in } (s, *) :\iff \forall a, b, c \in s\ (a \sim b \implies a*c \sim b*c)$$
}

\subsection{ Congruenza}
\bottom{
    Sia \(s \neq \emptyset\) e siano \(*_1, \dots, *_n\) operazioni binarie interne ad esso, e sia \(\sim\) una relazione di equivalenza.

    $$\sim \text{ congruenza in } (s, *_1, \dots, *_n) :\iff$$ $$\forall a, b, c, d \in s$$
    $$\forall i \in \mathbb N$$
    $$0 \leq i \leq n \implies (a \sim b \land c \sim d \implies a *_i c \sim b *_i d)$$
}

\subsection{ Epimorfismo fra Strutture e Strutture Quoziente}
\bottom{
    Se \(\sim\) è una congruenza in \((s, *_1, \dots, *_n)\), allora sono ben poste le operazioni $$\forall i: 0 \leq i \leq n:$$
    $$(*_i)_\sim: ([x]_\sim, [y]_\sim) \in s/\mathord\sim \times s/\mathord\sim \mapsto [x *_i y]_\sim \in s/\mathord\sim$$

    E la proiezione canonica \(\pi: x \in s \mapsto [x]_\sim \in s/\mathord\sim\) è epimorfismo tra le strutture \((s, *_1, \dots, *_n)\) e \((s/\mathord\sim, (*_1)_\sim, \dots, (*_n)_\sim)\)
}

\subsection{ Congruenza equivale a Compatibilità}
\bottom{
    Una relazione di equivalenza su una struttura \((s, *_1, \dots, *_n)\) è una\\
    congruenza $\iff$ è compatibile con ogni operazione della struttura.
}
\bottomp{
    Posso supporre un'unica operazione in \((s, *)\).

    \(\Rightarrow)\) Se ipotizziamo 
    $$a, b \in s: a \sim b$$ 
    $$c\in s: c \sim c$$
    abbiamo $$(c*a \sim c*b) \land (a*c \sim b*c)$$
    \(\Leftarrow)\) Supponiamo che \(a \sim b \land c \sim d\).\\
    Per ipotesi di compatibilità a destra
    $$a*c \sim b*c$$
    per compatibilità a sinistra
    $$b*c \sim b*d$$
    abbiamo $$a*c \sim b*c \sim b*d$$
    quindi per transitività
    $$a*c \sim b*d$$
}

\subsection{ Anello Quoziente di \(\mathbb Z\)}
\bottom{
    Sia \(\equiv_m\) una congruenza in \((\mathbb Z, +, \cdot)\). Allora essa è epimorfa all'anello quoziente (\(\mathbb Z_m, +_m, \cdot_m)\).
}

\subsection{ Asserti Equivalenti su Anelli Quoziente di \(\mathbb Z\)}
\bottom{
    Sia \(m \in \mathbb Z - \{0\}.\) Sono equivalenti le seguenti affermazioni:
\begin{enumerate}
    \item \((\mathbb Z_m, +_m, \cdot_m)\) è un campo.
    \item \((\mathbb Z_m, +_m, \cdot_m)\) è un dominio di integrità.
    \item \(m\) è primo.
\end{enumerate}
}

\bottomp{
    \(1 \rightarrow 2)\) Ovvio perché ogni campo è dominio di integrità.

    \(2 \rightarrow 3)\) Siano \(a, b \in \mathbb Z: m = ab\). Allora $$[m]_m = [0]_m = [ab]_m = [a]_m \cdot [b]_m$$ Trovandoci in un dominio di integrità, vale la Legge di Annullamento del Prodotto $$[a]_m \cdot [b]_m = [0]_m \iff [a]_m = [0]_m \lor [b]_m = [0]_m$$
    Suppongo senza ledere la generalità che \([a]_m = [0]_m\), cioè $$a \equiv_m 0 \iff (\exists k \in \mathbb Z)(a = km)$$ Allora $$m = ab = kmb \implies kb = 1 \implies b = \pm 1 \land a = \pm m$$ Allora \(m\) è irriducibile in \(\mathbb Z\), ed è quindi anche primo.

    \(3 \rightarrow 1)\) Sia \([a]_m \neq [0]_m\). Posso scegliere \(0 < a < |m|\). \(m\) è irriducibile, cioè i suoi divisori sono \(\{\pm 1, \pm m\}\) quindi \(MCD(a, m) = \{\pm 1\}\) e per il Teorema di Bézout $$\exists u, v \in \mathbb Z\ (1 = au + mv)$$
    Allora si ha che $$[1]_m = [au + mv]_m = [au]_m + [0]_m = [au]_m = [a]_m \cdot [u]_m$$ quindi \([a]_m\) è invertibile.\\
    Dato che ogni elemento è invertibile ci troviamo in un Campo.
}

\subsection{ Equazione Diofantea}
\bottom{
    Siano \(a, b, c \in \mathbb Z\). La funzione:

    $$e[a, b, c]: (x, y) \in \mathbb Z \times \mathbb Z \mapsto ax + by - c \in \mathbb Z$$

    Si dice equazione diofantea di 1° grado, a due incognite, con termini a, b e c.
}

\subsection{ Notazione Sintetica di Equazione Diofantea}
\bottom{
    Un'equazione diofantea della forma \(e[a,b,c](x,y)\) si può esprimere sinteticamente come \(ax + by = c\)
}

\subsection{ Soluzione dell'Equazione Diofantea}
\bottom{
    Data un'equazione diofantea \(e[a, b, c](m, n)\), la coppia \((x, y)\) per cui $$e[a, b, c](x, y) = 0 \iff ax + by = c$$ si dice, se esiste, soluzione dell'equazione diofantea.
}

\subsection{ Asserti Equivalenti al Teorema di Bézout}
\bottom{
    Siano \(a, b \in \mathbb Z, d \in MCD(a, b)\). Allora sono equivalenti i seguenti:
    \begin{enumerate}
        \item Il Teorema di Bézout
        \item \(a, b\) coprimi \(\iff (\exists u, v \in \mathbb Z)(1 = au + bv)\)
        \item \(<a, b> = d\mathbb Z^*\)
        \item L'equazione diofantea \(ax + by = c\) ha soluzioni \(\iff d|c\)
    \end{enumerate}
    $^* d\mathbb Z=\{dn\mid n\in \mathbb Z\}$ è l'insieme di tutti i numeri interi che possono essere ottenuti moltiplicando $d$ per un numero intero.\\
}
\bottomp{
    \(1 \rightarrow 2)\) \(\rightarrow)\) Per Bézout, se \(a, b\) sono coprimi, allora esistono $$u, v: 1 = au + bv$$
    \(\leftarrow)\) Se \(\exists u,v: 1 = au + bv \implies d|1\) perchè i divisori comuni dividono ogni combinazione lineare degli elementi (Proprietà di Divisione Lineare dei Divisori Comuni). Ma allora \(d = \pm 1\), e quindi \(a, b\) sono coprimi.

    \(2 \rightarrow 3)\) \(a, b \in d\mathbb Z\) e \(d\mathbb Z\) sottogruppo, quindi \(<a,b> \subseteq d\mathbb Z\). Scrivo $$a = a_1d \land b = b_1d$$ Poiché \(d \in MCD(a, b)\), \(a_1, b_1\) sono coprimi. Allora, per (2), trovo $$u, v \in \mathbb Z: 1 = a_1u + b_1v \implies d = a_1du + b_1dv = au + bv \in <a, b>$$
    Per asimmetria dunque \(d\mathbb Z = <a, b>\).

    \(3 \rightarrow 4)\) \(\rightarrow)\) Ci sono $$m,n \in \mathbb Z: am + bn = c$$ $$d | a \land d|b \implies d|c$$
    \(\leftarrow)\) Sia \(d|c\), allora \(c \in d\mathbb Z = <a, b>\) per (3). Ma allora $$\exists m,n \in \mathbb Z\ (am + bn = c)$$
    \(4 \rightarrow 1)\) Se prendo \(d = c\), ha soluzioni \(ax + by = d\), cioè $$\exists m,n \in \mathbb Z\ (am + bn = d)$$ che è esattamente Bézout.
}

\subsection{ Caratterizzazione dell'Insieme delle Soluzioni di Equazioni Diofantee}
\bottom{
    Sia \(ax + by = c\) un'equazione diofantea con soluzione \((x_0, y_0)\). Allora se \(d \in MCD(a, b)\), l'insieme delle soluzioni dell'equazione è $$\{(x_0 + \frac b d k, y_0 - \frac a d k) \mid k \in \mathbb Z\}$$
}
\bottomp{
    Chiamiamo l'insieme delle soluzioni \(s\), e l'insieme che vogliamo dimostrare equivalente \(m\), per comodità. Vogliamo dunque dimostrare che \(m = s\).

    \(\subseteq)\)Sostituendo si vede che $$m \subseteq s:
    a(x_0 + \frac b d k) + b(y_0 - \frac a d k) = ax_0 + by_0 + \cancel{\frac{ab}{d}k} - \cancel{\frac{ab}{d}k} = c$$

    \(\supseteq)\) Sia \((x, y) \in s\). Cioè, \(ax + by = c = ax_0 + by_0\).
    Allora $$a(x - x_0) = b(y_0 - y) \implies \frac a d (x-x_0) = \frac b d (y_0 - y)$$ dato che $d \in MCD(a, b)$

    \(\frac a d, \frac b d\) sono coprimi, quindi per il lemma di Euclide:
    $$\exists h, k \in \mathbb Z: \begin{cases}
    h\frac a d = y_0 - y \\
    k\frac b d = x - x_0
    \end{cases}$$

    E quindi, sostituendo:

    $$
    \frac a d (k \frac b d) = \frac b d (h \frac a d) \implies h = k \implies x = x_0 + k \frac b d \land y = y_0 - k \frac a d
    $$
    Da cui la tesi.
}

\subsection{ Equazione Congruenziale}
\bottom{
    \(m \in \mathbb Z - \{0\}, a, b \in \mathbb Z\). Allora la funzione:

    $$ec[a, b, m]: [n]_m \in \mathbb Z_m \mapsto [an - b]_m \in \mathbb Z_m$$

    Si dice equazione congruenziale di 1° grado, con una incognita, di termini a e b e modulo m.
}

\subsection{ Soluzione dell'Equazione Congruenziale}
\bottom{
    \(n \in \mathbb Z\) si dice soluzione di un'equazione congruenziale \(ec[a, b, m]\) se \(ec[a, b, m](n) = [0]_m\), ovvero se \(an \equiv_m b\).
    Chiaramente, dalla definizione di classi di resto, si ottiene che ogni valore congruente ad \(n\), cioè appartenente a \([n]_m\), è a sua volta soluzione.
}

\subsection{ Equazione Congruenziale come Equazione Diofantea}
\bottom{
    Un'equazione congruenziale \(ax \equiv_m b\) si può esprimere nella forma:
    $$ax + my = b$$
    
    Cioè come equazione diofantea.
}
\bottomp{
    $$ax \equiv_m b \implies m|(ax - b) \implies \exists k \in z\ (mk = ax - b)$$
    E quindi:
    $$ax - mk = b$$
    Ponendo \(y = -k\) abbiamo la tesi.
}

\subsection{ Criterio per l'Esistenza di Soluzioni di un'Equazione Congruenziale}
\bottom{
    Siano \(a, b \in \mathbb Z, m \in \mathbb Z - \{0\}, d \in MCD(a, m)\). Allora \(ax \equiv_m b\) ha soluzioni \(\iff d|b\).
}
\bottomp{
    L'equazione congruenziale \(ax \equiv_m b\) può essere espressa come equazione diofantea \(ax + my = b\). Per la $4^a$ Tesi di "Teorema sugli Asserti Equivalenti al Teorema di Bézout", allora l'equazione diofantea ha soluzioni solo se \(d \in MCD(a, m)\) divide \(b\).
}

\subsection{ Primo Corollario del Teorema sull'Esistenza di Soluzioni di un'Equazione Congruenziale}
\bottom{
    Siano \(a, b \in \mathbb Z, m \in \mathbb Z - \{0\}, d \in MCD(a, m)\). Allora:

    $$[a]_m \in U(\mathbb Z_m) \iff a,m \text{ coprimi}$$
}
\bottomp{
    \(\rightarrow)\) Esiste \([u]_m\) tale che \([a]_m \cdot [u]_m = [1]_m\). Ma quindi l'equazione congruenziale \(ax \equiv_m 1\) ha soluzione \(u\), e questo implica (per il Teorema sull'Esistenza di Soluzioni) che \(d|1\). Ma \(1|d\) quindi i due sono associati e \(1\) è MCD, quindi \(a\) e \(m\) sono coprimi.

    \(\leftarrow\) Se \(a,m\) sono coprimi, allora \(1\) è MCD e ovviamente \(1|1\) quindi per il Teorema sull'Esistenza delle Soluzioni, l'equazione congruenziale \(ax \equiv_m 1\) ha soluzioni e quindi esiste un \([a]_m\) è invertibile.
}

\subsection{ Secondo Corollario del Teorema sull'Esistenza di Soluzioni di un'Equazione Congruenziale}
\bottom{
    Siano \(a, b \in \mathbb Z, m \in \mathbb Z - \{0\}, d \in MCD(a, m)\). Allora:

    $$[a]_m \in U(\mathbb Z_m) \iff [a]_m \text{ non è divisore dello zero}$$
}
\bottomp{
    \(\rightarrow)\)Per assurdo, sia \([a]_m\) divisore dello zero. Allora $$\exists [b]_m \in \mathbb Z_m - \{[0]_m\}: [a]_m[b]_m = [0]_m$$ Ma invertibilità implica cancellabilità, e quindi si avrebbe \([b]_m = [0]_m\) il che è assurdo.

    \(\leftarrow)\) Per assurdo, sia \([a]_m\) non invertibile. Allora per il 1° Corollario, \(a, m\) non sono coprimi. Allora prendo \(d \in \mathbb Z\) con \(d \neq 1\) tale che $$\exists k \in \mathbb Z\ (ad = km)$$ Quindi:
    $$[a]_m[d]_m = [ad]_m = [km]_m = [0]_m
    $$ che è assurdo.
}

\subsection{ Risoluzione di Eq. Congruenziali: Termini Congruenti}
\bottom{
    \(a, b \in \mathbb Z, m \in \mathbb Z - \{0\}\).

    L'equazione congruenziale \(ax \equiv_m b\) ha lo stesso insieme di soluzioni dell'equazione congruenziale \(a'x \equiv_m b'\), per ogni  \(a' \in [a]_m, b' \in [b]_m\).
}
\bottomp{
    Dato che \([a']_m = [a]_m \land [b']_m = [b]_m\) per ipotesi:
    $$ax \equiv_m b \iff [a]_m[x]_m = [b]_m \iff a'x \equiv_m b'x$$
}



\subsection{ Risoluzione di Eq. Congruenziali: Equazione "Multiplo"}
\bottom{
    \(a, b \in \mathbb Z, m \in \mathbb Z - \{0\}\).

Osserviamo che, nel risolvere \(ax \equiv_m b\), allora \(\forall k \in \mathbb Z - \{0\}\) si ha che l'equazione \(akx \equiv_{mk} bk\) ha lo stesso insieme di equazioni.
}
\bottomp{
    Questo deriva semplicemente dal fatto che:
    $$ax + my = b \iff akx + mky = bk$$

    E da questo segue che se abbiamo che $$\exists k \in \mathbb Z\ (a = a'k \land b = b'k \land m = m'k)$$ allora l'equazione \(a'x \equiv_{m'} b'\) ha lo stesso insieme di soluzioni di \(ax \equiv_m b\).
}

\subsection{ Risoluzione di Eq. Congruenziali: Coprimi del Modulo}
\bottom{
    \(a, b \in \mathbb Z, m \in \mathbb Z - \{0\}\).

    Per ogni \(k\) coprimo ad \(m\), l'equazione \(akx \equiv_m bk\) ha lo stesso insieme di soluzioni di \(ax \equiv_m b\).
}
\bottomp{
    Sia \(x\) soluzione dell'equazione congruenziale \(akx \equiv_m bk\).
    Quindi $$[a]_m[k]_m[x]_m = [b]_m[k]_m$$
    Dato che \(k\) è coprimo ad \(m\), \([k]_m\) è invertibile e dunque cancellabile.
}

\subsection{ Algoritmo per la Soluzione di Equazioni Congruenziali}
\bottom{
    Data \(ax \equiv_m b\), allora per risolverla seguiamo i seguenti step:
    \begin{enumerate}
        \item Ridurre \(a, b\) in modo tale che \(0 \leq a, b \leq m - 1\).
        \item Prendere \(d \in MCD(a,m)\). Se \(d \nmid b\), non ho soluzioni. Se \(d|b\), continuo.
        \item Scrivo \(a = a'd, b = b'd, m = m'd\). Passo all'equazione equivalente \(a'x \equiv_{m'} b'\).
        \item Trovo l'inverso (in \((\mathbb Z_{m'}, \cdot)\)) di \([a']_{m'}\), tramite l'algoritmo delle divisioni successive esteso, e lo dico \([k]_{m'}\).
        \item L'insieme delle soluzioni è \([b'k]_{m'}\), poiché è una classe di resto di modulo m/d con \(d \in MCD(a,m)\)
    \end{enumerate}
}

\subsection{ Elemento Periodico di un Gruppo}
\bottom{
    Sia \((g, \cdot)\) un gruppo, \(x \in g\) si dice periodico se:

    $$\exists n \in \mathbb N - \{0\}\ (x^n = 1_g)$$
}

\subsection{ Periodo di un Elemento Periodico}
\bottom{
    Siano \((g, \cdot)\) un gruppo ed $ x \in g$ un suo elemento periodico. $$\text{Il minimo } n \in \mathbb N - \{0\}: x^n = 1_g$$ si dice periodo di \(x\) e si indica come \(|x|\).
}

\subsection{ Relazione fra Periodo e Cardinalità del Sottogruppo Generato}
\bottom{
    Siano $(g, \cdot)$ un gruppo ed $x \in g$ un elemento periodico. Allora:

    \(|x| = n \iff |<x>| = n\)

    Cioè il periodo di \(x\) è uguale alla cardinalità del sottogruppo generato da \(x\).
}

\subsection{ Teorema su Esponenti di Elementi Periodici e Congruenza}
\bottom{
    Siano $(g, \cdot)$ un gruppo ed $x \in g$ un elemento periodico, tale che \(|x| = m \in \mathbb N - \{0\}\). Allora:
    $$\forall a, b \in \mathbb Z\ (x^a = x^b \iff a \equiv_m b)$$
}
\bottomp{
    Sia \(x^a = x^b\). Moltiplichiamo ambo i membri per l'inverso di \(x^b\) e abbiamo quindi che
    $$x^a = x^b \iff x^a \cdot x^{-b} = x^b \cdot x^{-b} = x^{b-b} = 1_g \iff x^{a-b} = 1_g = x^0$$
    
    Prendiamo \(DE(a - b, m) = (q, r)\). Quindi $$1_g = x^{a-b} = x^{qm + r} = (x^m)^q \cdot x^r = (1_g)^q \cdot x^r = x^r$$
    Dato che \(0 \leq r < m\), allora \(r = 0\) e quindi \(a \equiv_m b\).
}
