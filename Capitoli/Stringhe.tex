\section{Stringhe}
\subsection{Insieme delle Stringhe}
\bottom{
    Scriviamo: \(\mathbb Z_2 = \mathbb Z / \equiv_2 = \{[0]_2, [1]_2\}\).
    Allora, dato \(n \in \mathbb N\), definiamo \(a = \mathbb Z_2 \times ... \times \mathbb Z_2 \text{ (} n \text{ volte)}\) l'insieme delle stringhe di 0 ed 1 di lunghezza \(n\).
}

\subsection{Somma e Prodotto Puntuali di Stringhe}
\bottom{
    Sia \(a = \mathbb Z_2 \times ... \times \mathbb Z_2\), \(n \in \mathbb N\) volte. Allora, presi \(x, y \in a\) tali che \(x = (x_1, ..., x_n), y = (y_1, ..., y_n), x_i, y_i \in \{0, 1\} \forall i \in I_n\), allora definiamo la somma puntuale:
    $$x + y = ([x_1 + y_1], [x_2 + y_2], ..., [x_n + y_n])$$
    Analogamente, definiamo il prodotto puntuale:
    $$x \cdot y = ([x_1 \cdot y_1], ..., [x_n \cdot y_n])$$
}

\subsection{Stringhe e Funzioni Caratteristiche}
\bottom{
    Dato un insieme \(s: |s| = n > 0\) e \(t \subseteq s\), allora la funzione caratteristica \(\chi_{t, s}\) può essere vista analoga ad una stringa, dato che associa ad ogni elemento di \(t\) o 0 o 1.
}

\subsection{L'Anello delle Parti e l'Anello delle Stringhe sono Isomorfi}
\bottom{
    Sia \(n \in \mathbb N - \{0\}\) e \(s = \{1, 2, ..., n\}\). Allora la funzione $$\varphi: x \in P(s) \mapsto \chi_{t, s} \in \mathbb Z_2 \times ... \mathbb Z_2$$ (\(n\) volte) è un'isomorfismo fra $$(P(s), \triangle, \cap)$$ e $$(\mathbb Z_2 \times ... \times \mathbb Z_2 \text{ (n volte)}, +, \cdot)$$ (dove \(+, \cdot\) sono la somma e prodotto puntuali).
}